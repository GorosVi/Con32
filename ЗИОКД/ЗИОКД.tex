\documentclass[a4paper,12pt]{report}
\usepackage{../GS6}

\begin{document}

	\def \nocredits {}
	\def \LineE {Конспект по дисциплине}
	\def \LineF {Защита и обработка конфиденциальных данных}

	\maketitle



%12.03.15

	Заколдаев Данил Анатольевич
	% Место работы: Морские Компьютерные системы

	Внезапные КР!

	Курсовик + презентация

	Лабы

	Определение документа
	Классификация
	Документационое обспечение управления
	Реквизиты

	\term{Документ} ГОСТ => Зафиксированная на материальном носителе инфрмация с реквизитами, позволящими её идентифицировать

	Классификация
		Содержеание
			Науч-тех
			Правовые
			Управленческие
		Времени создания
			Первичные
			Вторичные
		Способ изгомтовления
			Рукопсные
			машинописные
			На машносителе
			Фонодокумент
			Кино(видео) докумены

		Вид зафиксированной информации
			Письменные

		По способу чтения

		ПО месту издания

		ПО направлению отправления/ получения


		Виды классификации
			Иерархичская
			Фассетная => Для КР

		RTFM: ГОСТ 511 41-98

		Док Обесп Управление (ДОУ) Делопроищзводство- отрасль для документирования и организации работы с бумагой

		Составные счасти ДОУ
			Документиование деятельности
			lost

			Архивирование
			Уничтожение

			Документация - совокупнось документов в единыхх требованиях к оформленибю

			Система документации




		Реквизит:
			Обязательный элемент оформления официального документа
			Постоянная часть - Бланк,
			Переменая часть - номер письма,


		Реквизиты из ГОСТа 6.30/2003- Унифициованные СД
		\begin{enumerate}
		\item	Герб РФ
		\item	Герб субъекта РФ
		\item	Эмблема организации
		\item	Код организации
		\item	ОГРН
		\item	ИНН/КПП
		\item	Код формы док-та
		\item	Наименование орг-ии
		\item	Справочные данные об орг-ии
		\item	Наименование вида док-та
		\item	Дата док0та
		\item	Рег NСсылка на рег N
		\item	Место составления или издания
		\item	Адресат 15
		\item	Гри
		\end{enumerate}



		Пример оформления реквизитов - на слйдах

		Номер _всегда_ ставится от руки (без электронного документооборота)
		Если вхо не имеет N - ставим Б/Н



		Копия и дубликат

		Копия - полное воспроизведение информации. ено не имеет юридическрой силы

		Дубликат -  заверенная копия документа (реквизит копия верна).

		Заверение скриншота - нотариат.


		Почерковедческая экспертиза (не менее 3 слов) - на подпись вероятность 50/50
		Учитывается время создания документа и порядок простановки (сначала подпись, затем печать)

		Копии заверяются руководителем или уполномоченными лицами (ВРИО - временно исполняющий обязанность - определяется приказом по предприятию, с субъектом и сроком) (ИО - исп обязанность)

		Выписка из документа - фрагмент, напр, протокола



	Документооборот - движение документов в организации с момента создания


		Объём - характеристика документооборота

		Классификация, будучи правильно поставленной, не допускает появлнения избыточного документа.

		Виды обеспечения
			Техническое (железо, линии связи)
			Организационное (положения, реглаенты, распоряжение)
			Программное (софт)
			Информационное (Сама информация)



		Принципы организации документооборота
			Минимизация пути и времени документа
			Путь характеризуется количеством вершин, время - весами связей

			Централизация операция по приёмоотправке и первичной обработки
			Сокращение числа инстанций и исключение возвратных движений документа.
				Вощвратные движения возникают обычно когда прдразделения занимаются несвойственными им функциями

			Маршрутиззация
			Однократность регистрации
			Организация предварительного рассмотрения документов
			Вынесение конкретных резолюций
			lost


		Электронный документ - и-ия на матноситее ждя передачи в пространстве с использованием СВТ и электросвязи с целью хранения и общественного испрользоваия



	Принципы органиации документооборота



%12.03.15






\end{document}