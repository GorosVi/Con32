\documentclass[a4paper,12pt]{report}
\usepackage{GS6}

\begin{document}

	\def \nocredits {}
	\def \LineE {Конспект по дисциплине}
	\def \LineF {Философия}

	\maketitle



		% 1  Алексеенко

		% 5  Ананевич
		% 4  Брынцев
		% 3  Валявский
		% 2  Вершинин
		% 1  Горбатов

		% 5  Горошков
		% 4  Ермолаев
		% 3  Жидков
		% 2  Житков К
		% 1  Истомин

		% 5  Кашицин
		% 4  Коновалов
		% 3  Левкович
		% 2  Левчук => подойти
		% 1  Лукин

		% 5  Макоев
		% 4  Оболенсуий
		% 3  Пронькина
		% 2  Самойленко
		% 1  Седышева

		% 5  Сергеев
		% 4  Слепова
		% 3  Терлецкий
		% 2  Фролов
		% 1  Хан



%13.03.15

	Татьяна Алеесеева Новолоцкая

	% m_nov_lodn@mail.ru

	новолоцкая Т.А.

	Очерки по истории философии >> библиотека (методичка)

	Домашние КР: тексты по вариантам

	Next Платон "Миф о пещере"
	+ учебник ( Милецкая школа - Фалес, Анаксеман; Гераклит Парменит о познании; атеистическая модель Демокрита)
	Будет хитрый экспресс - опрос


	Консультации - предлагаться будут вопросы не с семинаров

	Среда, НЕЧ+ЧЁТ> после еминара в $17:^{20}$, длительность 1 час

	Домашняя работа сдаётся в печатном виде, объём 1 лист.

	Декарт, Ницше, Бердяев, Ортега + Бодриаро
	Платон,


	Лекция в ПТ


	1 модуль Онтология и гносиология

	1 лекция:
	Проблемы бытия и познания в античной философии
		1. Мировоззрние и его историчскеие типы
		2. Предмет философии. Функции. Структура и методы философского знания.
		3. Концепции бытия древнегреческой натурфилософии (Милецкая школа и античные атомизмы)
		4. Учение Гераклита и ПАрменида о быии и познании

	Мировоззрение - система взглядов человека на мир с целью определить своё отношение к миру и узнать своё место

	3 формы мировоззрения -
		Мифологическое
		Религиозное
		Философское


	Лосев - миф есть в словах данная чудесная личностная история.

	В тот период человек неразделен с коллективом

	Титаны, циклопы - внушающие человеку страх.
	Космос - в пер. с др. гр. порядок.

	В случае, когда человек не может определить причинно-следственные связи - они описывают это чудом.

		Миф о 12 подвиге Геракла.

		В мифологическом сознании нет разделение между словом и действительностью.



	Разложение мифологического сознания









	Религиозное мировоззрение









	Философское мировоззрение

	PDV: Философия началась и кончилась в Греции

	Философия - знание рациональное

	Материализм - первичность материи, природы

	Идеалисты - первичность разума

	Сферы философии
	\begin{enumerate}
		\item	Онтология - учение о бытии
		\item	Гносеология - учение о познании
		\item	Методология -
	\end{enumerate}

	Диалектика и метафизика
	ДЛ видит мир в процессе, развитии - мир как процесс.

	Метафизика - каждое явление уникально.

	Формирование двух направлений - рационализм и эмпиризм.


	Детерминизм - жёсткий (механистический)  - действует причина - появляется следствие

	Диалектический  - представлен в рассуждениях Маркса - ПСС - случайность есть форма проявления случайности
	Вероятностный детерминизм

	Антроплогия
	Социальная философия - наука об обществе

	Аксиология - учение о ценностях

	Агностики - отрицающие возможность иметь знания

	%~ lost

%25.03.15


Чтение - декарт1 и Декарт 2
Опрос - основные положения средневековой онтологии и гносиологии
	Бекен Новое время, 
%	учение о субстанции филосоыо 17 веа Спиноза декарт, Ньютон - НКМ, 
	Пузансуия Бруно




%27.03.15 - лекция

	Субстанцией у Платона является идея - сущность вещей (парадигма для вещи)
	
	Абстракция языка науки.
	

	\subsubsection{Отличия антологии Платона от антологии Аристотеля}
	[эксперимент тогда не был развит]	
	
	Автор логики
	
	
	4 век до Н.Э << - эллинистическая культура	
	Меняется представление человек о 
	
	Пиррон - (365-275 <<)
	3 вопроса: 
		Из чего состоят вещи
		Как мы должны к ним относиться к ним
		Какой результат мы получим от этого отношения
	
	Ответы Пиррона
		Мы не можем получить ответа
		
		
		
Средние века

	Тертулиан - Верую ибо абсурдно
	[нАгорная проповедь]
	
	Средневековая философия
		Основа - монотеизм
		Теоцентризм
		В основе христианской антологии лежит креационизм
		В основе гносиологии лежит ревеляционизм - истина даётся в откровении
		
		
	Аврелий Августин -  верую, потому и знаю
	
	Абеля - знаю, потому  верую/ Реабилитация знания	
	
	Схоласты 
		Номинализм (сначала существуют единичные вещи, унверсалии -  имена)
		Реалисты (сначала существует общее)
		
	Фома Аквинский - неотомизм - на земле правы номиналисты, до вещей - правы реалисты [вначале было слово - общее]
	Из метафизики Аристотеля - определение основных понятий науки - причина, элемент, ... . 2 общие причины - действующие и целевые - используются для доказательства 

	Эпоха Возрождения [14-16 ВВ]
	Петрарка "Моя тайна"
	
	
	Антиклерикализм - кружки
	
	Основным принципом эпохи является принцип антропоцентризма - на земле центром является человек
	
	Н. Кузанский 
	Пантеизм -вс1 есть Бог
	Эсхатология - учение о конце света
	
	
	
	Галилей - 1564 - 1652
	
	Философия 
	
	ТЗ декарта  - для самост отработки
	
	
	НКМ Ньютона
	
	Субстанциальная модель пространства и времени
	
	
	Немецкая классическая философия
	
	Кант Либерфихте
	Шеллинг
	Гегель
	Бах
	
	
	Вещь-в-себе - трансцендентна - находится по ту сторону нашего познания
	
	Наука ориентируется на вещь-для-нас - то, как мир нам является
	априорное (доопытное) = трансцедентальное (отличие от трансцендентной)
	
	Априорные формы чувственности - пространство и время
	
	Рассудочная деятельность - априорные категории мышления - способность мыслить то, что представляют чувства.
	
	Разум направлен не на опыт, а на создания правил для деятельности рассудка
		идеи
		Психологическая (о душе как безусловном единстве всех душ процессов
			Космологическя о мире как единстве
			Теологическая  - о безусловной причины всего сущего 
			
		
	


\end{document}